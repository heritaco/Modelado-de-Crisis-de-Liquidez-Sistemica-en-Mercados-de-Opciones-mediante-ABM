%=================================================================
% Presentación Beamer: Crisis de Liquidez en un Mercado de Opciones
%=================================================================
\documentclass[10pt, aspectratio=169]{beamer}

%--------------------------------------------------
% Idioma y fuentes básicas
%--------------------------------------------------
\usepackage[utf8]{inputenc}
\usepackage[T1]{fontenc}
\usepackage[spanish]{babel}
\usepackage{graphicx}
\usepackage{booktabs}

\graphicspath{{../figuras/}}

%--------------------------------------------------
% Tema Beamer (puedes cambiarlo a Madrid, Berlin, etc.)
%--------------------------------------------------
\usepackage{fontspec}
\setmainfont{EB Garamond}[
  UprightFont = * Regular,
  ItalicFont = * Italic,
  BoldFont = * SemiBold,
  BoldItalicFont = * SemiBold Italic,
]
\setsansfont{EB Garamond}[
  UprightFont = * Regular,
  ItalicFont = * Italic,
  BoldFont = * SemiBold,
  BoldItalicFont = * SemiBold Italic,
]
\usetheme{metropolis}
\usecolortheme{default}
\definecolor{primarycolor}{HTML}{0b0a40}
\setbeamercolor{title}{fg=primarycolor}

\setbeamercolor{structure}{fg=blue!70!black}
\setbeamercolor{palette primary}{bg=primarycolor,fg=white} 
\setbeamercolor{palette secondary}{bg=blue!60!black,fg=white}
\setbeamercolor{palette tertiary}{bg=blue!50!black,fg=white}
\setbeamercolor{background canvas}{bg=white}


\usecolortheme{default}

% Quitar símbolos de navegación
\setbeamertemplate{navigation symbols}{}

% Pie de página sencillo con título abreviado
\setbeamertemplate{footline}{
  \leavevmode%
  \hbox{
    \begin{beamercolorbox}[wd=.7\paperwidth,ht=2.5ex,dp=1ex,leftskip=1em]{author in head/foot}
      \usebeamerfont{author in head/foot}\insertshorttitle
    \end{beamercolorbox}%
    \begin{beamercolorbox}[wd=.3\paperwidth,ht=2.5ex,dp=1ex,rightskip=1em]{date in head/foot}
      \usebeamerfont{date in head/foot}\insertframenumber{} / \inserttotalframenumber
    \end{beamercolorbox}
  }%
}

%--------------------------------------------------
% Datos de la portada
%--------------------------------------------------
\title[ABM Opciones y Liquidez]{Crisis de Liquidez en un Mercado de Opciones: \\ Un Modelo Basado en Agentes}
\author[H.\ Espino Montelongo]{Los $\pi$ mosqueteros}
\institute[UDLAP]{Universidad de las Américas Puebla}
\date{\today}

%--------------------------------------------------
% Mostrar una diapositiva al inicio de cada sección
%--------------------------------------------------
\AtBeginSection[]
{
  \begin{frame}
    \centering
    \vfill
    \begin{beamercolorbox}[sep=10pt,center,rounded=true,shadow=true]{title}
      \usebeamerfont{title}\insertsectionhead\par
    \end{beamercolorbox}
    \vfill
  \end{frame}
}

%=================================================================
\begin{document}
%=================================================================

%--------------------------------------------------
\begin{frame}
  \titlepage
\end{frame}

%--------------------------------------------------
\begin{frame}{Contenido}
  \tableofcontents
\end{frame}

%=================================================================
\section{Motivación y objetivos}
%=================================================================

%--------------------------------------------------
\begin{frame}{Motivación}
  \begin{itemize}
    \item Los mercados de opciones son clave para:
      \begin{itemize}
        \item Cobertura de riesgos
        \item Especulación
        \item Formación de precios
      \end{itemize}
    \item En episodios de estrés pueden aparecer:
      \begin{itemize}
        \item Ventas forzadas y \emph{margin calls} en cadena
        \item Ampliación brusca de \emph{spreads}
        \item Desaparición temporal de liquidez
      \end{itemize}
    \item Estos fenómenos dependen de:
      \begin{itemize}
        \item Apalancamiento
        \item Reglas de margen
        \item Estructura de red entre participantes
      \end{itemize}
  \end{itemize}
\end{frame}

%--------------------------------------------------
\begin{frame}{¿Por qué usar modelos basados en agentes (ABM)?}
  \begin{itemize}
    \item Modelos clásicos: suelen asumir un agente representativo y mercados líquidos.
    \item Pero en crisis importa:
      \begin{itemize}
        \item Quién está apalancado
        \item Quién provee liquidez
        \item Cómo se conectan los balances
      \end{itemize}
    \item Un ABM permite:
      \begin{itemize}
        \item Representar tipos de agentes con reglas distintas
        \item Observar cómo surgen cascadas y contagio \emph{desde abajo}
        \item Probar “qué pasaría si…” cambiamos márgenes, apalancamiento, etc.
      \end{itemize}
  \end{itemize}
\end{frame}

%--------------------------------------------------
\begin{frame}{Objetivo de la presentación}
  \begin{itemize}
    \item Mostrar un modelo sencillo de mercado de opciones basado en agentes.
    \item Enseñar, con ejemplos visuales, cómo:
      \begin{itemize}
        \item Un shock de precio puede generar una crisis de liquidez.
        \item Los traders apalancados amplifican el choque.
        \item La red de agentes se fragmenta en plena crisis.
      \end{itemize}
    \item Discutir qué nos dice el modelo sobre:
      \begin{itemize}
        \item Límites de apalancamiento
        \item Políticas de margen
        \item Medición de riesgo sistémico
      \end{itemize}
  \end{itemize}
\end{frame}

%=================================================================
\section{Descripción intuitiva del modelo}
%=================================================================

%--------------------------------------------------
\begin{frame}{Mapa general del modelo}
  \begin{block}{Mercado simulado}
    \begin{itemize}
      \item Un solo activo subyacente $S_t$ (precio de una acción).
      \item Opciones call europeas at-the-money sobre ese activo.
      \item Tiempo discreto: un “día” por paso de simulación.
    \end{itemize}
  \end{block}
  \begin{block}{Qué se simula en cada día}
    \begin{enumerate}
      \item Movimiento del precio del subyacente (con posibilidad de shock).
      \item Recalcular precio de la opción y volatilidad implícita.
      \item Decisiones de los agentes y ejecuciones en el mercado.
      \item Actualizar capital, estrés y posibles \emph{margin calls}.
    \end{enumerate}
  \end{block}
\end{frame}

%--------------------------------------------------
\begin{frame}{Tipos de agentes}
  \begin{columns}
    \column{0.48\textwidth}
    \textbf{Market Makers}
    \begin{itemize}
      \item Proveen cotizaciones bid/ask.
      \item Gestionan inventarios de opciones.
      \item A más volatilidad o inventario, más ancho el \emph{spread}.
    \end{itemize}
    \vspace{0.3cm}
    \textbf{Especuladores}
    \begin{itemize}
      \item Reglas simples de:
        \begin{itemize}
          \item \emph{Momentum}: seguir la tendencia.
          \item Reversión: apostar a que el precio vuelve.
        \end{itemize}
      \item Operan si el movimiento del día supera cierto umbral.
    \end{itemize}
    \column{0.48\textwidth}
    \textbf{Hedgers}
    \begin{itemize}
      \item Tienen un portafolio grande de acciones.
      \item Usan opciones para cubrir una fracción del portafolio.
      \item En crisis aumentan su nivel de cobertura.
    \end{itemize}
    \vspace{0.3cm}
    \textbf{Traders apalancados}
    \begin{itemize}
      \item Capital propio pequeño, exposición grande.
      \item Sujetos a reglas de margen.
      \item Si el capital cae demasiado, sufren \emph{margin calls} y liquidaciones.
    \end{itemize}
  \end{columns}
\end{frame}

%--------------------------------------------------
\begin{frame}{Dinámica del precio y del mercado}
  \begin{itemize}
    \item El precio del subyacente se mueve como un proceso aleatorio “suave”.
    \item En un día específico se introduce un shock (por ejemplo, \textbf{-10\%}).
    \item A partir de ahí:
      \begin{itemize}
        \item El precio de la opción cae.
        \item La volatilidad implícita sube (el mercado exige mayor prima de riesgo).
      \end{itemize}
    \item El modelo marca el régimen “CRISIS” cuando:
      \begin{itemize}
        \item Ocurre el shock, o
        \item Se acumulan suficientes \emph{margin calls}.
      \end{itemize}
  \end{itemize}
\end{frame}

%--------------------------------------------------
\begin{frame}{Reglas clave de comportamiento}
  \begin{block}{Market Makers}
    \begin{itemize}
      \item Si tienen demasiado inventario o la volatilidad sube:
        \begin{itemize}
          \item Ensanchan sus \emph{spreads} para protegerse.
          \item En crisis, el \emph{spread} se puede duplicar.
        \end{itemize}
    \end{itemize}
  \end{block}
  \begin{block}{Especuladores}
    \begin{itemize}
      \item Se activan sólo si el movimiento del día supera su umbral.
      \item \emph{Momentum}: compran cuando el precio ya subió.
      \item Reversión: venden cuando el precio subió demasiado.
    \end{itemize}
  \end{block}
\end{frame}

%--------------------------------------------------
\begin{frame}{Reglas clave de comportamiento (II)}
  \begin{block}{Hedgers}
    \begin{itemize}
      \item Cada cierto número de días ajustan su cobertura.
      \item En crisis:
        \begin{itemize}
          \item Aumentan el porcentaje cubierto.
          \item Generan demanda adicional de opciones justo cuando hay menos liquidez.
        \end{itemize}
    \end{itemize}
  \end{block}
  \begin{block}{Traders apalancados}
    \begin{itemize}
      \item Comparan su capital con el valor de la posición.
      \item Si el colchón de capital es demasiado pequeño:
        \begin{itemize}
          \item Sufren una \emph{margin call}.
          \item El modelo los liquida y registra una nueva liquidación.
        \end{itemize}
      \item Varias de estas liquidaciones en cadena es lo que llamamos “cascada”.
    \end{itemize}
  \end{block}
\end{frame}

%--------------------------------------------------
\begin{frame}{Escenario base}
  \begin{itemize}
    \item Un shock de precio moderado (por ejemplo, -10\%) en el paso 50.
    \item Población típica:
      \begin{itemize}
        \item 7 market makers
        \item 40 especuladores
        \item 15 hedgers
        \item 25 traders apalancados
      \end{itemize}
    \item Volatilidad inicial: 20\%.
    \item Plazo de la opción: 30 días (se va reduciendo con el tiempo).
    \item Variamos:
      \begin{itemize}
        \item Nivel de apalancamiento promedio.
        \item Cantidad de market makers.
        \item Momento y magnitud del shock.
      \end{itemize}
  \end{itemize}
\end{frame}

%=================================================================
\section{Resultados visuales}
%=================================================================

%--------------------------------------------------
\begin{frame}{Panel de estado del mercado}
  \begin{columns}
    \column{0.5\textwidth}
      \begin{itemize}
        \item En el ejemplo mostrado:
          \begin{itemize}
            \item \textbf{Estado}: CRISIS
            \item Precio subyacente $\approx\$91$ (desde \$100)
            \item Volatilidad implícita $\approx 23\%$ (desde 20\%)
            \item 22 \emph{margin calls} acumuladas
          \end{itemize}
        \item Con los deslizadores podemos:
          \begin{itemize}
            \item Cambiar el número de agentes de cada tipo.
            \item Mover el paso del shock.
            \item Ajustar la magnitud del shock.
          \end{itemize}
      \end{itemize}
    \column{0.5\textwidth}
      % Reemplaza el nombre de archivo
      \includegraphics[width=0.6\linewidth]{estado actual.png}
  \end{columns}
\end{frame}

%--------------------------------------------------
\begin{frame}{Red de agentes y contagio}
  \begin{columns}
    \column{0.55\textwidth}
      \includegraphics[width=\linewidth]{network.png}
    \column{0.45\textwidth}
      \begin{itemize}
        \item Cada punto es un agente.
        \item El color indica su “salud”:
          \begin{itemize}
            \item Verde: sano
            \item Naranja: estresado
            \item Negro: liquidado
          \end{itemize}
        \item En el ejemplo:
          \begin{itemize}
            \item La red se rompe en dos grandes grupos.
            \item Varios nodos negros se concentran en un clúster.
            \item Esto refleja que el mercado deja de estar bien conectado.
          \end{itemize}
      \end{itemize}
  \end{columns}
\end{frame}

%--------------------------------------------------
\begin{frame}{Espacio de fases: precio vs volatilidad}
  \begin{columns}
    \column{0.55\textwidth}
      \includegraphics[width=\linewidth]{phase space analysis.png}
    \column{0.45\textwidth}
      \begin{itemize}
        \item Cada punto es un día: precio en el eje X, volatilidad en Y.
        \item Inicio: alrededor de (100, 20\%).
        \item Shock: salto brusco a menor precio y mayor volatilidad.
        \item Después del shock:
          \begin{itemize}
            \item El sistema ya no vuelve a la región de baja volatilidad.
            \item Se queda en un “nuevo régimen” más inestable.
          \end{itemize}
      \end{itemize}
  \end{columns}
\end{frame}

%--------------------------------------------------
\begin{frame}{Cascadas de liquidación y estrés}
  \begin{columns}
    \column{0.6\textwidth}
      \includegraphics[width=\linewidth]{liquidation cascades.png}
    \column{0.4\textwidth}
      \textbf{Arriba: liquidaciones acumuladas}
      \begin{itemize}
        \item Curva plana antes del shock.
        \item Saltos en escalones después del shock.
        \item Cada escalón = nueva ola de \emph{margin calls}.
      \end{itemize}
      \vspace{0.3cm}
      \textbf{Abajo: estrés promedio}
      \begin{itemize}
        \item Apalancados: estrés cerca de 1 (casi todos al límite).
        \item Hedgers: aumentan su estrés al subir cobertura.
        \item MM y especuladores: estrés bajo en promedio.
      \end{itemize}
  \end{columns}
\end{frame}

%--------------------------------------------------
\begin{frame}{Distribuciones de salud y capital}
  \begin{columns}
    \column{0.6\textwidth}
      \includegraphics[width=\linewidth]{statistical distributions.png}
    \column{0.4\textwidth}
      \textbf{Salud de los agentes}
      \begin{itemize}
        \item Mayoría cerca de 1 (capital casi intacto).
        \item Cola izquierda: algunos agentes muy dañados.
        \item Los liquidados se acumulan en la parte más baja.
      \end{itemize}
      \vspace{0.3cm}
      \textbf{Capital por tipo}
      \begin{itemize}
        \item Hedgers: los más grandes.
        \item Market makers: intermedios.
        \item Especuladores: pequeños.
        \item Apalancados: capital promedio bajo y golpeado por la crisis.
      \end{itemize}
  \end{columns}
\end{frame}

%--------------------------------------------------
\begin{frame}{Volatilidad a lo largo del tiempo}
  \begin{columns}
    \column{0.6\textwidth}
      \includegraphics[width=\linewidth]{spectral analysis.png}
    \column{0.4\textwidth}
      \textbf{Serie temporal}
      \begin{itemize}
        \item Antes del shock: volatilidad estable cerca del 20\%.
        \item Después: sube y se mantiene en una banda más alta.
      \end{itemize}
      \vspace{0.3cm}
      \textbf{En frecuencia (FFT)}
      \begin{itemize}
        \item La mayor parte de la energía está en frecuencias bajas.
        \item La volatilidad responde más a movimientos de “medio plazo” que a ruido muy rápido.
      \end{itemize}
  \end{columns}
\end{frame}

%--------------------------------------------------
\begin{frame}{Índice de contagio y fragilidad}
  \begin{columns}
    \column{0.6\textwidth}
      \includegraphics[width=\linewidth]{contagion n systimic risk.png}
    \column{0.4\textwidth}
      \textbf{Índice de contagio}
      \begin{itemize}
        \item Mide qué proporción de agentes está seriamente dañada.
        \item En este ejemplo, el contagio fuerte se concentra en apalancados.
      \end{itemize}
      \vspace{0.3cm}
      \textbf{Fragilidad de red}
      \begin{itemize}
        \item Captura cuán distintos son los niveles de estrés.
        \item Picos grandes: coexisten agentes casi intactos con otros muy dañados.
      \end{itemize}
  \end{columns}
\end{frame}

%=================================================================
\section{Limitaciones, extensiones y conclusiones}
%=================================================================

%--------------------------------------------------
\begin{frame}{Limitaciones principales}
  \begin{itemize}
    \item Libro de órdenes agregado:
      \begin{itemize}
        \item No modelamos cada nivel de precio ni cada orden límite.
      \end{itemize}
    \item Reglas de comportamiento sencillas:
      \begin{itemize}
        \item No hay algoritmos de alta frecuencia ni estrategias demasiado sofisticadas.
      \end{itemize}
    \item Un solo subyacente y una sola opción:
      \begin{itemize}
        \item No hay contagio entre activos o clases de instrumentos.
      \end{itemize}
    \item Parámetros estilizados:
      \begin{itemize}
        \item No está calibrado contra datos reales de posiciones.
      \end{itemize}
  \end{itemize}
\end{frame}

%--------------------------------------------------
\begin{frame}{Extensiones futuras}
  \begin{itemize}
    \item \textbf{Microestructura más rica}
      \begin{itemize}
        \item Libro de órdenes explícito.
        \item Impacto de mercado según tamaño de la orden.
      \end{itemize}
    \item \textbf{Red de crédito y cámara de compensación}
      \begin{itemize}
        \item Exposiciones entre instituciones.
        \item Márgenes dinámicos según volatilidad y concentración.
      \end{itemize}
    \item \textbf{Aprendizaje y adaptación}
      \begin{itemize}
        \item Agentes que modifican sus reglas si pierden dinero.
      \end{itemize}
    \item \textbf{Calibración empírica}
      \begin{itemize}
        \item Ajustar parámetros para imitar episodios históricos (2008, 2020, etc.).
      \end{itemize}
  \end{itemize}
\end{frame}

%--------------------------------------------------
\begin{frame}{Conclusiones}
  \begin{itemize}
    \item Un shock moderado puede volverse una crisis de liquidez si:
      \begin{itemize}
        \item Hay mucho apalancamiento.
        \item Las reglas de margen son laxas.
      \end{itemize}
    \item Los traders apalancados son el principal canal de amplificación:
      \begin{itemize}
        \item \emph{Margin calls} y liquidaciones en cadena.
        \item Aumentos de volatilidad y fragmentación de la red.
      \end{itemize}
    \item Los ABM ayudan a pensar en diseño de políticas:
      \begin{itemize}
        \item Límites de apalancamiento.
        \item Esquemas de margen más prudentes.
        \item Métricas de riesgo sistémico basadas en redes y estrés.
      \end{itemize}
  \end{itemize}
\end{frame}



%=================================================================
\end{document}
%=================================================================
